\setchapterpreamble[u]{\margintoc}
\chapter{Basic Commands in Linux}
\labch{basic}

\section{Navigating the File System}
The file system can be navigated in the Linux command line using the following commands:
\begin{itemize}
  \item \textbf{pwd}: Print the current working directory
  \item \textbf{ls}: List the contents of the current directory
  \item \textbf{cd}: Change the current working directory
  \item \textbf{mkdir}: Create a new directory
  \item \textbf{rmdir}: Remove a directory
  \item \textbf{touch}: Create a new file
  \item \textbf{rm}: Remove a file
  \item \textbf{pushd}: Push the current directory to a stack
  \item \textbf{popd}: Pop the current directory from a stack\sidenote{\textbf{pushd} and \textbf{popd} are useful for quickly switching between directories in scripts.}
\end{itemize}

More details about these commands can be found in their respective
man pages. For example, to find more about the \textbf{ls} command,
you can type \texttt{man ls}

\begin{qs}
  What is the command to list the contents of the current directory?
\end{qs}

\begin{ans}
  \texttt{ls}
\end{ans}


\begin{qs}
  What is the command to list the contents of the current directory
  including hidden files?
\end{qs}

\begin{ans}
\texttt{ls -a}
\end{ans}

\begin{qs}
  What is the command to list the contents of the current directory
  in a long list format? (show permissions, owner, group, size, and time)
\end{qs}

\begin{ans}
\texttt{ls -l}
\end{ans}

\begin{qs}
  What is the command to list the contents of the current directory
  in a long list format and also show hidden files?
\end{qs}

\begin{ans}
\texttt{ls -al} or \texttt{ls -la} or \texttt{ls -l -a} or \texttt{ls -a -l}
\end{ans}

\begin{qs}
  The output of \texttt{ls} gives multiple files and directories in a single
  line. How can you make it print one file or directory per line?
\end{qs}

\begin{ans}
  \texttt{ls -1}\\
  This can also be done by passing the output of \texttt{ls} to \texttt{cat}
  or storing the output of \texttt{ls} in a file and then using \texttt{cat}
  to print it. We will see these in later weeks.\sidenote{that is a one, not an L}
\end{ans}

\section{File Permissions}
\begin{qs}
  How to list the permissions of a file?
\end{qs}

\begin{ans}
  \texttt{ls -l} \\
  The permissions are the first 10 characters of the output.\\
  \texttt{stat -c \%A filename} will list only the permissions of a file.\\
  There are other format specifiers of \texttt{stat} to show different statistics
  which can be found in \texttt{man stat}.
\end{ans}

\begin{qs}
  How to change permissions of a file?
  Let's say we want to change \texttt{file1}'s permissions to \texttt{rwxr-xr--}
  What is the octal form of that?
\end{qs}

\begin{ans}
  \texttt{chmod u=rwx,g=rx,o=r file1} will change the permissions of \texttt{file1}\\
  The octal form of \texttt{rwxr-xr--} is 754.\\
  So we can also use \texttt{chmod 754 file1}\\
  Providing the octal is same as using \texttt{=} to set the permissions.\\
  We can also use \texttt{+} to add permissions and \texttt{-} to remove permissions.
\end{ans}

\section{Inodes and Links}

\begin{qs}
  How to list the inodes of a file?
\end{qs}

\begin{ans}
  \texttt{ls -i} will list the inodes of a file.
  The inodes are the first column of the output of \texttt{ls -i}
  This can be combined with other flags like \texttt{-l} or \texttt{-a} to show more details.
\end{ans}

\begin{qs}
  How to create soft link of a file?
\end{qs}

\begin{ans}
  \texttt{ln -s sourcefile targetfile} will create a soft link of \texttt{sourcefile}
  named \texttt{targetfile}.
  The soft link is a pointer to the original file.
\end{ans}

\begin{definition}[Soft Links]
\labdef{softlinks}
  Soft Links are special kinds of files that just store the path
  given to them. Thus the path given while making soft links should
  either be an absolute path, or relative \textbf{from} the location of the
  soft link \textbf{to} the location of the original file. It should not be
  relative from current working directory.\footnote{This is a common mistake.}
\end{definition}

\begin{qs}
  How to create hard link of a file?
\end{qs}

\begin{ans}
  \texttt{ln sourcefile targetfile} will create a hard link of \texttt{sourcefile}
  named \texttt{targetfile}. The hard link is same as the original file. It does
  not depend on the original file anymore after creation. They are equals,
  both are \texttt{hardlinks} of each other. There is no parent-child relationship.
  The other file can be deleted and the original file will still work.
\end{ans}

\begin{definition}[Hard Links]
  Hard Links are just pointers to the same inode. They are the same file.
  They are not pointers to the path of the file. They are pointers to the
  file itself. They are not affected by the deletion of the other file.
  When creating a hard link, you need to provide the path of the original
  file, and thus it has to be either absolute path, or relative from the
  current working directory, not relative from the location of the hard link.
\end{definition}

\begin{qs}
  How to get the real path of a file?\\
  Assume three files:
  \begin{itemize}
    \item \textbf{file1} is a soft link to \textbf{file2}
    \item \textbf{file2} is a soft link to \textbf{file3}
    \item \textbf{file3} is a regular file
  \end{itemize}
  Real path of all these three should be the same. How to get that?
\end{qs}

\begin{ans}
  \texttt{realpath filename} will give the real path of \texttt{filename}. \\
  You can also use \texttt{readlink -f filename} to get the real path.
\end{ans}

\section{System Management and Information}

\begin{qs}
  How to print the current date and time in some custom format?
\end{qs}

\begin{ans}
  \texttt{date -d today +\%Y-\%m-\%d} will print the current date in the format
  \texttt{YYYY-MM-DD}. The format can be changed by changing the format specifiers.
  The format specifiers are given in the \texttt{man date} page. The \texttt{-d today} can
  be dropped, but is mentioned to show that the date can be changed to any date.
  It can be strings like '2024-01-01' or '5 days ago' or 'yesterday', etc.
\end{ans}

\begin{qs}
  How to print the kernel version of your system?
\end{qs}

\begin{ans}
  \texttt{uname -r} will print the kernel version of your system.
  \texttt{uname} is a command to print system information.
  The \texttt{-r} flag is to print the kernel release.
  There are other flags to print other system information. \\
  We can also run \texttt{uname -a} to get all fields and extract only the
  kernel info using commands taught in later weeks.
\end{ans}

\begin{qs}
  How to see how long your system is running for? \\
  What about the time it was booted up?
\end{qs}

\begin{ans}
  \texttt{uptime} will show how long the system is running for.\\
  \texttt{uptime -s} will show the time the system was booted up. \\
  The \texttt{-s} flag is to show the time of last boot.
\end{ans}

\begin{qs}
  How to see the amount of free memory? What about free hard disk space?
  If we are unable to understand the big numbers, how to convert them to human readable format?
  What is difference between MB and MiB?
\end{qs}

\begin{ans}
  \texttt{free} will show the amount of free memory. \\
  \texttt{df} will show the amount of free hard disk space. \\
  \texttt{df -h} and \texttt{free -h}
  will convert the numbers to human readable format. \\
  MB is Megabyte, and MiB is Mebibyte. \\
  1 MB = 1000 KB, 1 GB = 1000 MB, 1 TB = 1000 GB, this is SI unit. \\
  1 MiB = 1024 KiB, 1 GiB = 1024 MiB, 1 TiB = 1024 GiB, this is $2^{10}$ unit.
\end{ans}

\section{Reading and Writing Files using cat}

\begin{qs}
  Can we print contents of multiple files using a single command?
\end{qs}

\begin{ans}
  \texttt{cat file1 file2 file3} will print the contents of \texttt{file1}, \texttt{file2}, and \texttt{file3}
  in the order given. The contents of the files will be printed one after the other.
\end{ans}

\begin{qs}
  Can \texttt{cat} also be used to write to a file?
\end{qs}

\begin{ans}
  Yes, \texttt{cat > file1} will write to \texttt{file1}. The input will be taken from the
  terminal and written to \texttt{file1}. The input will be written to \texttt{file1} until
  the user presses \texttt{Ctrl+D} to indicate end of input.
  This is \texttt{redirection}, which we see in later weeks.
\end{ans}

\begin{qs}
  \texttt{ls} can only show files and directories in the \textbf{cwd}\sidenote{\textbf{cwd} means \textbf{Current Working Directory}}, not subdirectories.
  True or False?
\end{qs}

\begin{ans}
  False. \texttt{ls} can show files and directories in the cwd, and also in subdirectories.
  The \texttt{-R} flag can be used to show files and directories in subdirectories, recursively.
\end{ans}

\section{Types of Files}

\begin{qs}
  What types of files are possible in a linux file system?
\end{qs}

\begin{ans}
  There are 7 types of files in a linux file system:
  \begin{itemize}
    \item Regular Files (starts with \texttt{-})
    \item Directories (starts with \texttt{d})
    \item Symbolic Links (starts with \texttt{l})
    \item Character Devices (starts with \texttt{c})
    \item Block Devices (starts with \texttt{b})
    \item Named Pipes (starts with \texttt{|})
    \item Sockets (starts with \texttt{s})
  \end{itemize}
\end{ans}

\begin{qs}
  How to know what kind of file a file is? Can we determine using
  its extension? Can we determine using its contents? What does
  \textit{
  \href{https://developer.mozilla.org/en-US/docs/Web/HTTP/Basics_of_HTTP/MIME_types}{MIME} mean?
}
How to get that?
\end{qs}

\begin{ans}
  The \texttt{file} command can be used to determine the type of a file. \\
  The extension of a file does not determine its type. \\
  The contents of a file can be used to determine its type. \\
  MIME stands for Multipurpose Internet Mail Extensions. \\
  It is a standard that indicates the nature and format of a document. \\
  \texttt{file -i filename} will give the MIME type of \texttt{filename}.
\end{ans}
\section{Types of Commands}

\begin{qs}
  How to create aliases? How to make them permanent? How to unset them?
\end{qs}

\begin{ans}
  \texttt{alias name='command'} will create an alias. \\
  \texttt{unalias name} will unset the alias. \\
  To make them permanent, add the alias to the \texttt{$\sim$/.bashrc} file. \\
  The \texttt{$\sim$/.bashrc} file is a script that is executed whenever a new terminal is opened.
\end{ans}

\begin{qs}
  How to run the normal version of a command if it is aliased?
\end{qs}

\begin{ans}
  \texttt{\textbackslash command} will run the normal version of \texttt{command} if it is aliased.
\end{ans}

\begin{qs}
  What is the difference between \texttt{which}, \texttt{whatis}, \texttt{whereis}, \texttt{locate}, and \texttt{type}?
\end{qs}

\begin{ans}
  Each of the commands serve a different purpose:
  \begin{itemize}
    \item \texttt{which} will show the path of the command that will be executed.
    \item \texttt{whatis} will show a short description of the command.
    \item \texttt{whereis} will show the location of the command, source files, and man pages.
    \item \texttt{locate} is used to find files by name.
    \item \texttt{type} will show how the command will be interpreted by the shell.
  \end{itemize}
\end{ans}

\begin{exercise}
  Find the path of the \texttt{true} command using \texttt{which}.
  Find a short description of the \texttt{true} command using \texttt{whatis}.
  Is the executable you found actually the one that is executed when you run \texttt{true}? Check using \texttt{type true}
\end{exercise}

\begin{definition}[Types of commands]
  A command can be an alias, a shell built-in, a shell function, keyword, or an executable.
  The \texttt{type} command will show which type the command is.
  \begin{itemize}
    \item \textbf{alias}: A command that is an alias to another command defined
    by the user or the system.
    \item \textbf{builtin}: A shell built-in command is a command that is built
    into the shell itself. It is executed internally by the shell.
    \item \textbf{file}: An executable file that is stored in the file system.
      It has to be stored somewhere in the \textbf{PATH} variable.
    \item \textbf{function}: A shell function is a set of commands that are
    executed when the function is called.
    \item \textbf{keyword}: A keyword is a reserved word that is part of the shell
    syntax. It is not a command, but a part of the shell syntax.
  \end{itemize}
\end{definition}

\section{File Manipulation}

\begin{qs}
  How to list only first 10 lines of a file? How about first 5? Last 5?
  How about lines 105 to lines 152?
\end{qs}

\begin{ans}
  \texttt{head filename} will list the first 10 lines of \texttt{filename}. \\
  \texttt{head -n 5 filename} will list the first 5 lines of \texttt{filename}. \\
  \texttt{tail -n 5 filename} will list the last 5 lines of \texttt{filename}. \\
  \texttt{head -n 152 filename | tail -n 48} will list lines 105 to 152 of \texttt{filename}.
  This uses \texttt{|} which is a pipe, which we will see in later weeks.
\end{ans}

\begin{qs}
  Do you know how many lines a file contains? How can we count it?
  What about words? Characters?
\end{qs}

\begin{ans}
  \texttt{wc filename} will count the number of lines, words, and characters in \texttt{filename}. \\
  \texttt{wc -l filename} will count the number of lines in \texttt{filename}. \\
  \texttt{wc -w filename} will count the number of words in \texttt{filename}. \\
  \texttt{wc -c filename} will count the number of characters in \texttt{filename}.
\end{ans}

\begin{qs}
  How to delete an empty directory? What about non-empty directory?
\end{qs}

\begin{ans}
  \texttt{rmdir dirname} will delete an empty directory. \\
  \texttt{rm -r dirname} will delete a non-empty directory.
\end{ans}

\begin{qs}
  How to copy an entire folder to another name? What about moving? \\
  Why the difference in flags?
\end{qs}

\begin{ans}
  \texttt{cp -r sourcefolder targetfolder} will copy an entire folder to another name. \\
  \texttt{mv sourcefolder targetfolder} will move an entire folder to another name. \\
  The difference in flags is because \texttt{cp} is used to copy, and \texttt{mv} is used to move or rename a file or folder.
  The \texttt{-r} flag is to copy recursively, and is not needed for \texttt{mv} as it is not recursive
  and simply changes the name of the folder (or the path).
\end{ans}
