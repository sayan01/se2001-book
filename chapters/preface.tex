\chapter*{Preface}
\addcontentsline{toc}{chapter}{Preface} % Add the preface to the table of contents as a chapter

Through this work I have tried to make learning and
understanding the basics of Linux fun and easy. I have
tried to make the book as practical as possible, with
many examples and exercises. The structure of the book
follows the structure of the course \textit{BSSE2001
- System Commands}, taught by
\textbf{
\href{https://home.iitm.ac.in/gphani/}{Prof. Gandham Phanikumar}
}
at
\textbf{
\href{https://study.iitm.ac.in/ds/}{IIT Madras BS Program}.
}.

The book takes inspiration from the previous works done for the course,

\begin{itemize}
	\item \href{https://github.com/prassr}{Sanjay Kumar}'s
	\href{https://github.com/prassr/gnu-linux-commands}{Github Repository}
	\item \href{https://github.com/cheriangeorge}{Cherian George}'s
	\href{https://github.com/prassr/gnu-linux-commands}{Github Repository}
	\item \href{https://github.com/prabuddhmathur}{Prabuddh Mathur}'s
	\href{https://www.youtube.com/playlist?list=PLLuZiiAWg2wpaEOBWAVO7Jl35MQkOyi6-}{TA Sessions}
\end{itemize}

The book covers basic commands, their motivation, use cases, and examples.
The book also covers some advanced topics like shell scripting, regular expressions, and text processing using \texttt{sed} and \texttt{awk}.

This is not a substitute for the course, but a companion to it.
The book is a work in progress and any contribution is welcome at
\url{https://github.com/sayan01/se2001-book}

\begin{flushright}
	\textit{Sayan Ghosh}
\end{flushright}
